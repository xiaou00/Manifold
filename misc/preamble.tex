\usepackage{pgfplots}
\usepackage{tikz}
\usepackage{silence}
\usepackage[most]{tcolorbox}
\WarningFilter{hyperref}{}
\WarningFilter{latexfont}{}
\usepackage{tikz-cd}
\usepackage{amsmath}
\usepackage{amssymb}
\usepackage{amsthm}
\usepackage{amsfonts}
\usepackage{mathrsfs}
\usepackage{titling}
\usepackage{pgfplots}
\usepackage[hidelinks]{hyperref}
\usepackage{xcolor}
\usepackage{ulem}
\usepackage{quiver}
\usepackage{titlesec}
\usepackage{pdfpages}
\usepackage{quiver}
\usepackage{mathptmx}
\usepackage{bbm}
\usepackage[a4paper, left=2.5cm, right=2.5cm, top=2.5cm, bottom=2cm]{geometry}

\setlength{\droptitle}{-3cm}
\pgfplotsset{compat=1.18}
\pagestyle{empty}
\theoremstyle{definition}
\renewcommand{\contentsname}{}

%chktex-file 37
%chktex-file 35
%chktex-file 25
%chktex-file 18
%chktex-file 17
%chktex-file 13
%chktex-file 11
%chktex-file 9
%chktex-file 8
%chktex-file 3
%chktex-file 1

\renewcommand\proofname{\indent Pf.}

\newtheorem{remark}{注.}

%标题样式

\newenvironment{comm}[1][]{
    \begin{tcolorbox}[
        enhanced,
        breakable,
        colframe=gray,
        coltitle=gray,
        colback=white,
        colbacktitle=white,
        boxrule=1pt,
        titlerule=-1pt,
        title={\bf #1},
    ]

}{
    \end{tcolorbox}
}

\newenvironment{cirno}[1][]{
    \begin{tcolorbox}[
        enhanced,
        breakable,
        colframe=white,
        coltitle=cyan!75!gray,
        colback=cyan!75!gray!10!white,
        colbacktitle=cyan!75!gray!10!white,
        boxrule=1pt,
        titlerule=-1pt,
        title={\bf #1},
        overlay unbroken and last={
            \node[anchor=north east, xshift=0, yshift=0] at (frame.north east) {
                \includegraphics[width=1.2cm]{resources/cirno.png}
            };
        },
    ]

}{
    \end{tcolorbox}
}

\newenvironment{rumia}[1][]{
    \begin{tcolorbox}[
        enhanced,
        breakable,
        colframe=white,
        coltitle=orange!75!gray,
        colback=orange!75!gray!10!white,
        colbacktitle=orange!75!gray!10!white,
        boxrule=1pt,
        titlerule=-1pt,
        title={\bf #1},
        overlay unbroken and last={
            \node[anchor=north east, xshift=0, yshift=0] at (frame.north east) {
                \includegraphics[width=1.2cm]{resources/rumia.png}
            };
        },
    ]

}{
    \end{tcolorbox}
}

\newenvironment{dayousei}[1][]{
    \begin{tcolorbox}[
        enhanced,
        breakable,
        colframe=white,
        coltitle=green!75!blue!75!gray,
        colback=green!75!blue!75!gray!10!white,
        colbacktitle=green!75!blue!75!gray!10!white,
        boxrule=1pt,
        titlerule=-1pt,
        title={\bf #1},
        overlay unbroken and last={
            \node[anchor=north east, xshift=0, yshift=0] at (frame.north east) {
                \includegraphics[width=1.2cm]{resources/dayousei.png}
            };
        },
    ]

}{
    \end{tcolorbox}
}

\newcounter{theorem}[section]%=====定理=====
\renewcommand{\thetheorem}{{定理 }\thesection.\arabic{theorem}}
\newenvironment{theorem}[1][]{
    \refstepcounter{theorem}
    \begin{cirno}[\bf \thetheorem. #1]
    
}{
    \end{cirno}
}
%=====引理=====
\newcommand{\thelemma}{{引理 }\thesection.\arabic{theorem}}
\newenvironment{lemma}[1][]{
    \refstepcounter{theorem}
    \begin{cirno}[\bf \thelemma. #1]
    
}{
    \end{cirno}
}
%=====推论=====
\newcommand{\thecorollary}{{推论 }\thesection.\arabic{theorem}}
\newenvironment{corollary}[1][]{
    \refstepcounter{theorem}
    \begin{cirno}[\bf \thecorollary. #1]
    
}{
    \end{cirno}
}
%=====命题=====
\newcommand{\theproposition}{{命题 }\thesection.\arabic{theorem}}
\newenvironment{proposition}[1][]{
    \refstepcounter{theorem}
    \begin{cirno}[\bf \theproposition. #1]
    
}{
    \end{cirno}
}
%=====定义=====
\newcounter{definition}[section]
\renewcommand{\thedefinition}{{定义 }\thesection.\arabic{definition}}
\newenvironment{definition}[1][]{
    \refstepcounter{definition}
    \begin{rumia}[\bf \thedefinition. #1]
    
}{
    \end{rumia}
}
%=====例题=====
\newcounter{example}[section]
\renewcommand{\theexample}{{例 }\thesection.\arabic{example}}
\newenvironment{example}[1][]{
    \refstepcounter{example}
    \begin{dayousei}[\bf \theexample. #1]
    
}{
    \end{dayousei}
}
\newenvironment{solution}{
    \begin{proof}[Sol.]}{\end{proof}}

\title{微分流形}
\author{xiaou0}
\renewcommand{\O}{\varnothing}
\newcommand{\C}{\mathbb{C}}
\newcommand{\R}{\mathbb{R}}
\newcommand{\Z}{\mathbb{Z}}
\newcommand{\N}{\mathbb{N}}
\newcommand{\Q}{\mathbb{Q}}
\newcommand{\del}{\partial}
\renewcommand{\d}{\mathrm{d}}
\newcommand{\Hom}{\mathrm{Hom}}
\newcommand{\set}[1]{\left\{\,#1\,\right\}}
\newcommand{\ob}{\operatorname{ob}}
\newcommand{\xto}{\xrightarrow}
\newcommand{\hto}{\hookrightarrow}
\newcommand{\tto}{\twoheadrightarrow}

\newcommand{\Grp}{\mathsf{Grp}}
\newcommand{\Ab}{\mathsf{Ab}}
\newcommand{\Set}{\mathsf{Set}}
\newcommand{\Diff}{\mathsf{Diff}}
\newcommand{\Top}{\mathsf{Top}}